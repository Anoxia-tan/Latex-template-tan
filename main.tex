%ctexart/ctexbook/ctexrep: 这些是ctex 宏包提供的文档类,适用于排版中文文章、书籍和报告
\documentclass[a4paper,UTF8]{ctexart}
%newtxtext 是一个 LaTeX 宏包,用于提供一个新的 Times 字体风格,适用于文本排版。它是 newtx 宏包系列的一部分,提供了一套新的 Times 风格字体,包括正文字体、粗体、斜体等,以及一些调整,用于更好地支持中英文混排。
%\usepackage{newtxtext}
\usepackage[left=2.5cm, right= 2.5cm, top=3.0cm, bottom=3.2cm]{geometry}
\usepackage[dvipsnames,svgnames]{xcolor}
\usepackage[strict]{changepage} % 提供一个 adjustwidth 环境
\usepackage{framed} % 实现方框效果
\definecolor{formalshade}{rgb}{0.95,0.95,1} % 文本框颜色
%%%%%%%%%%%%%%%%%%%%%%%%%%%%%%%%%%%%%%%%%%%%%%%%%%%%%%%%%%%%%%%%%%%%%%%%%%%%%%%%
\usepackage{titlesec}
\titleformat{\section}[block]{\normalfont\Large\bfseries}{\thesection}{1em}{}
\titlespacing*{\section}{0pt}{*3.5}{*1.5}
%%%%%%%%%%%%%%%%%%%%%%%%%%%%%%%%%%%%%%%%%%%%%%%%%%%%%%%%%%%%%%%%%%%%%%%%%%%%%%%%
\usepackage{listings}

% ------------------******-------------------
% 注意行末需要把空格注释掉,不然画出来的方框会有空白竖线
\newenvironment{formal}{%
\def\FrameCommand{%
\hspace{1pt}%
{\color{DarkBlue}\vrule width 2pt}%
{\color{formalshade}\vrule width 4pt}%
\colorbox{formalshade}%
}%
\MakeFramed{\advance\hsize-\width\FrameRestore}%
\noindent\hspace{-4.55pt}% disable indenting first paragraph
\begin{adjustwidth}{}{7pt}%
\vspace{2pt}\vspace{2pt}%
}
{%
\vspace{2pt}\end{adjustwidth}\endMakeFramed%
}
% ------------------******-------------------
\begin{document}
\section{c++仿函数}
\begin{formal}
C++中的仿函数(Functor)是一种特殊的对象,它可以像函数一样被调用,但实际上是一个类的实例。仿函数可以通过重载函数调用运算符 operator() 来实现,从而使得该对象可以像函数一样被调用。仿函数主要用于在算法中作为操作符使用,例如在STL(标准模板库)中的各种算法中。通过使用仿函数,可以将操作抽象出来,使得算法更加灵活,可以适用于不同的数据类型或者自定义的操作。
\end{formal}
\begin{lstlisting}[language=C++, caption=My C++ Code, label=lst:cpp]
#include <iostream>

// 定义一个加法仿函数
class AddFunctor {
public:
    int operator()(int a, int b) {
        return a + b;
    }
};

int main() {
    AddFunctor adder;  // 创建仿函数对象

    int result = adder(3, 5);  // 使用仿函数调用
    std::cout << "Result:" << result << std::endl;

    return 0;
}

\end{lstlisting}

\end{document}
