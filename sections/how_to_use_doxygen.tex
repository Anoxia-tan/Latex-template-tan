\pagestyle{fancy}
\rhead{}
\lhead{}
\chead{instructions for doxygen}
\lfoot{}
\cfoot{-\ \thepage \  -}
\rfoot{}
%%%%%%%%%%%%%%%%%%%%%%%%%%%%%%%%%%%%%%%%%%%%
\textbf{Doxygen}可以根据我们的代码注释生成我们需要的文档,这个文档明显比自己整理的要好看的多有结构的多,另外它还能生成每个文件的调用关系图、每个类的继承关系图、每个函数的调用关系图、每个文件夹的包含关系图……
\section{项目注释}
\qquad 项目注释块用于对项目进行描述,每个项目只出现一次,一般可以放在main.c主函数文件头部。对于其它类型的项目,置于定义项目入口函数的文件中。对于无入口函数的项目,比如静态库项目,置于较关键且不会被外部项目引用的文件中。
\par 项目注释块以/** @mainpage开头,以*/结束。章节可自行添加。
\begin{lstlisting}[language=C++, label=lst:cpp]
	/**
	* @mainpage  项目名
	* @section   项目详细描述
	* @section   功能描述  
	* @section   用法描述 
	* @section   固件更新 
	*/
\end{lstlisting}
\section{文件注释}
\qquad 一般在每一个文件开头加入版权、作者、时间等描述。
文件注释描述了该文件的内容,如果一个文件只声明,或实现,或测试了一个对象,并且这个对象已经在它的声明处进行了详细的注释,那么就没必要再加上文件注释,除此之外的其他文件都需要文件注释。
\begin{lstlisting}[language=C++, label=lst:cpp]
    /**
    * @file 文件名
    * @brief 简介
    * @details 细节
    * @author 作者
    * @version 版本号
    * @date 年-月-日
    * @copyright 版权
    */
\end{lstlisting}
\section{类的注释}
\qquad 每个类的定义都要附带一份注释, 描述类的功能和用法,除非它的功能相当明显。
如果类的声明和定义分开了(例如分别放在了 .h 和 .cpp 文件中)。 此时,描述类用法的注释应当和接口定义放在一起,描述类的操作和实现的注释应当和实现放在一起。
\begin{lstlisting}[language=C++, label=lst:cpp]
    /**
    * @brief 类的详细描述
    */
\end{lstlisting}
\section{函数注释}
\qquad 基本上每个函数声明处前都应当加上注释, 描述函数的功能和用途,对一些必要参数进行解释说明。只有在函数的功能简单而明显时才能省略这些注释(例如, 简单的取值和设值函数)
\begin{lstlisting}[language=C++, label=lst:cpp]
	/**
	* @brief 函数描述
	* @param 参数描述
	* @return 返回描述
	* @retval 返回值描述
	*/
\end{lstlisting}
\section{常量/变量注释}
\qquad 通常变量名本身足以很好说明变量用途,某些情况下, 也需要额外的注释说明。
\subsection{常量/变量上一行注释} 
\begin{lstlisting}[language=C++, label=lst:cpp]
	/// ... 注释 ... 
	变量
	//! ... 注释 ... 
	变量
	/*! ... 注释 ... */
	变量
	/** ... 注释 ... */
	变量
\end{lstlisting}
\subsection{常量/变量后注释}
\begin{lstlisting}[language=C++, label=lst:cpp]
	变量 /*!< ... 注释 ... */
	变量 /**< ... 注释 ...*/
	变量 //!< ... 注释 ...
	变量 ///< ... 注释 ...
\end{lstlisting}
\section{枚举类型注释}
\qquad 枚举可看做类的一种特殊形式,添加简单说明,Doxygen会根据其后面的enum解析出枚举类型。
\begin{lstlisting}[language=C++, label=lst:cpp]
	/**
	* @brief 简要说明
	*
	* @details 详细说明
	*/
	enum 枚举类型名
	{
		值1,/**< ...值1的注释... */
		值2 /**< ...值2的注释... */
	};
\end{lstlisting}
\section{宏注释}
\begin{lstlisting}[language=C++, label=lst:cpp]
	/**
	* @brief 宏简要说明
	* 
	* @details 宏详细说明
	*/
	#define ... ...
\end{lstlisting}
\section{包含一段代码的注释}
\begin{lstlisting}[language=C++, label=lst:cpp]
	/// @code
	要包含的代码内容
	/// @endcode
\end{lstlisting}
\section{分组注释}
\qquad 表明注释块是一组类、文件或名称空间的记录。这可以用于对类、文件或名称空间进行分类,并记录这些类别。还可以将组用作其他组的成员,从而构建组的层次结构。
\begin{lstlisting}[language=C++, label=lst:cpp]
	/*! \defgroup mygrp This is my group.
	* 	这里为 mygrp 组添加记录
	* 	@{
		*/ 
		
		/*! A function */
		void func()
		{}
		
		/*! @} */
\end{lstlisting}
\section{特殊命令简介}
\begin{center}
	\begin{tabular}{|p{2cm}|p{5cm}|p{7cm}|} \hline % 其中,|c|表示文本居中,文本两边有竖直表线。
		\rowcolor{formalshade}
		\textbf{命令} & \textbf{说明} & \textbf{语法}  \\ \hline
		@author & 作者 & author \{ list of authors \}\\ \hline
		@brief &	简介&	brief \{ brief description \}\\ \hline
		@bug & 缺陷,链接到所有缺陷汇总的缺陷列表 & bug\{ description\}\\ \hline
		@copyright	&版权&	copyright \{ copyright description \}	\\ \hline
		@code .. @endcode &在注释中开始说明一段代码,直到@endcode命令结束& \\ \hline
		@date&	年-月-日 &	date \{ date description \}	 \\ \hline
		@deprecated	& 弃用说明。可用于描述替代方案,预期寿命等 &	deprecated { description }	\\ \hline
		@details &	细节 &	details \{ detailed description \}	\\ \hline
		@file & 文件名 & file [< name >]  \\ \hline
		@link &	连接(与@see类库,\{@link www.google.com\}) &	link < link-object>	\\ \hline
		@mainpage & 主页信息 & mainpage [(title)]\\ \hline
		@param &	参数&	param [(dir)] < parameter-name> \{ parameter description \}	\\ \hline
		@pre &用来说明代码项的前提条件&pre \{ description of the precondition \} \\ \hline
		@relates & 通常用做把非成员函数的注释文档包含在类的说明文档中 & relates [<name>] \\ \hline
		@return	&描述返回意义	& return \{ description of the return value \}	\\ \hline
		@retval	&描述返回值意义&	retval <return value> \{ description \}	\\ \hline
		@see &	参考 &	see \{ references \}	\\ \hline
		@since & 通常用来说明从什么版本、时间写此部分代码 & since\{ description \} \\ \hline
		@throw &	异常描述 &	throw < exception-object> \{ exception description \}	\\ \hline
		@todo &	待处理 &	todo \{ paragraph describing what is to be done \}	\\ \hline
		@version	&版本号 &	version \{ version number \}	\\ \hline
		@warning &	警告信息 &	warning \{ warning message \}	\\ \hline
	\end{tabular}
\end{center}
\begin{flushright}
	\fontsize{8pt}{5pt}
	以上仅列出部分命令,详情请查看doxygen官方文档\\$https://www.doxygen.nl/manual/commands.html$
\end{flushright}
