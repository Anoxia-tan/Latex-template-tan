
\pagestyle{fancy}
\rhead{}
\lhead{}
\chead{Hello-algo}
\lfoot{}
\cfoot{-\ \thepage \  -}
\rfoot{}
%---------------------------------------------------------%
\section{初识算法}
\subsection{ 算法无处不在}
当我们听到“算法”这个词时,很自然地会想到数学。然而实际上,许多算法并不涉及复杂数学,而是更多
地依赖于基本逻辑,这些逻辑在我们的日常生活中处处可见。
\par 在正式探讨算法之前,有一个有趣的事实值得分享:你已经在不知不觉中学会了许多算法,并习惯将它们应
用到日常生活中了。下面,我将举几个具体例子来证实这一点。
\begin{formal}
C++中的仿函数(Functor)是一种特殊的对象,它可以像函数一样被调用,但实际上是一个类的实例。仿函数可以通过重载函数调用运算符 operator() 来实现,从而使得该对象可以像函数一样被调用。仿函数主要用于在算法中作为操作符使用,例如在STL(标准模板库)中的各种算法中。通过使用仿函数,可以将操作抽象出来,使得算法更加灵活,可以适用于不同的数据类型或者自定义的操作。
\end{formal}
\begin{lstlisting}[language=C++, label=lst:cpp]
#include <iostream>

// 定义一个加法仿函数
class AddFunctor {
public:
    int operator()(int a, int b) {
        return a + b;
    }
};

int main() {
    AddFunctor adder;  // 创建仿函数对象

    int result = adder(3, 5);  // 使用仿函数调用
    std::cout << "Result:" << result << std::endl;

    return 0;
}
\end{lstlisting}